\documentclass{beamer}

\usepackage{amsthm}
\usepackage[utf8]{inputenc}
\usepackage[T1]{fontenc}
\usepackage[brazil]{babel}
\usepackage[export]{adjustbox}
\usepackage{listings}
\usepackage{fontspec}
\usepackage{color}

\definecolor{pblue}{rgb}{0.13,0.13,1}
\definecolor{pgreen}{rgb}{0,0.5,0}
\definecolor{pred}{rgb}{0.9,0,0}
\definecolor{pgrey}{rgb}{0.46,0.45,0.48}

\lstset{language=Java,
  showspaces=false,
  showtabs=false,
  breaklines=true,
  showstringspaces=false,
  breakatwhitespace=true,
  commentstyle=\color{pgreen},
  keywordstyle=\color{pblue},
  stringstyle=\color{pred},
  basicstyle=\ttfamily,
}


\usetheme{Madrid}
\usecolortheme{beetle}
\usefonttheme{professionalfonts}

\setmainfont{Oswald}

\lstset{basicstyle=\ttfamily,breaklines=true}
\beamertemplatenavigationsymbolsempty

\begin{document}

\selectlanguage{brazil}
\title[Recursão]{Recursão}
\author{Prof. Andrey Masiero}

\begin{frame}[plain,noframenumbering]
  \titlepage
\end{frame}

\begin{frame}[plain,noframenumbering]
  \frametitle{Agenda}
  \tableofcontents
\end{frame}

\section{Recursão}

\begin{frame}
	\frametitle{Recursão}
    \begin{itemize}[<+->]
        \item Basicamente, é quando uma função utiliza em sua definição ela própria;
        \item Ela é utilizada para dar uma definição finita a um conjunto que poder ser infinito;
        \item É muito utilizado em problemas de dividir e conquistar.
    \end{itemize}
\end{frame}

\begin{frame}
    \frametitle{Exemplo: Fatorial}
    O fatorial de um número $n$, definido por $n!$, que é o produto de todos os inteiros positivos menores ou igual a $n$.

    \begin{exampleblock}{A definição formal é:}
        \[
            n! = \prod_{k=1}^{n} k\ \ \ \forall n \in \mathbb{N}
        \]


        \[
            5! = 1 \times 2 \times 3 \times 4 \times 5 = 120
        \]
    \end{exampleblock}

    \begin{alertblock}{Caso especial}
        \[
            0! = 1
        \]
    \end{alertblock}
\end{frame}

\begin{frame}
	\frametitle{Fatorial Não Recursivo}
    \centering
    \lstinputlisting[language=Java]{src/nonrecursive.java}
\end{frame}

\begin{frame}
	\frametitle{Fatorial Recursivo}
    \centering
    \lstinputlisting[language=Java]{src/recursive.java}
\end{frame}

\section{Exercícios}

\begin{frame}
    \frametitle{Exercícios}
    Dado a sequência de Fibonacci, com a definição formal abaixo, faça:
    \[
        F(n) =
        \begin{cases}
            0, & \text{se } n = 0; \\
            1, & \text{se } n = 1; \\
            F(n - 1) + F(n - 2) & \text{outros casos}.
        \end{cases}
    \]
    \begin{enumerate}
        \item Implemente a sequência de Fibonacci não recursiva.
        \item Implemente a sequência de Fibonacci recursiva.
        \item Desenvolva um algoritmo que calcule a soma dos $N$ primeiros números.
        \item Desenvolva um algoritmo que calcule a soma dos elementos de um vetor.
        \item Desenvolva um algoritmo que calcule o produto dos elementos de um vetor.
        \item Teste os algoritmos em um programa principal.
    \end{enumerate}
\end{frame}

\section{Referências}

\begin{frame}
    \frametitle{Referências Bibliográficas}
    \begin{enumerate}
        \item Cormen, Thomas H., Charles E. Leiserson, Ronald L. Rivest, and Clifford Stein. ``Introduction to algorithms second edition.'' (2001).
        \item Tamassia, Roberto, and Michael T. Goodrich. ``Estrutura de Dados e Algoritmos em Java.'' Porto Alegre, Ed. Bookman 4 (2007).
        \item Ascencio, Ana Fernanda Gomes, and Graziela Santos de Araújo. ``Estruturas de Dados: algoritmos, análise da complexidade e implementações em JAVA e C/C++.'' São Paulo: Perarson Prentice Halt 3 (2010).
    \end{enumerate}
\end{frame}

\end{document}
