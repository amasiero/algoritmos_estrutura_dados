\documentclass{beamer}

\usepackage{amsthm}
\usepackage[utf8]{inputenc}
\usepackage[T1]{fontenc}
\usepackage[brazil]{babel}
\usepackage[export]{adjustbox}
\usepackage{listings}
\usepackage{fontspec}
\usepackage{color}

\definecolor{pblue}{rgb}{0.13,0.13,1}
\definecolor{pgreen}{rgb}{0,0.5,0}
\definecolor{pred}{rgb}{0.9,0,0}
\definecolor{pgrey}{rgb}{0.46,0.45,0.48}

\lstset{language=Java,
  showspaces=false,
  showtabs=false,
  breaklines=true,
  showstringspaces=false,
  breakatwhitespace=true,
  commentstyle=\color{pgreen},
  keywordstyle=\color{pblue},
  stringstyle=\color{pred},
  basicstyle=\ttfamily,
}


\usetheme{Madrid}
\usecolortheme{beetle}
\usefonttheme{professionalfonts}

\setmainfont{Oswald}

\lstset{basicstyle=\ttfamily,breaklines=true}
\beamertemplatenavigationsymbolsempty

\begin{document}

\selectlanguage{brazil}
\title[Buscas e BubbleSort]{Busca Sequencial, Busca Binária e Ordenação BubbleSort}
\author{Prof. Andrey Masiero}

\begin{frame}[plain,noframenumbering]
  \titlepage
\end{frame}

\begin{frame}[plain,noframenumbering]
  \frametitle{Agenda}
  \tableofcontents
\end{frame}

\section{Busca Sequencial}

\begin{frame}
    \frametitle{Busca Sequencial}
    \begin{center}
        \begin{table}
            \begin{tabular}{| p{0.25cm} | p{0.25cm} | p{0.25cm} | p{0.25cm} | p{0.25cm} | p{0.25cm} | p{0.25cm} | p{0.25cm} | p{0.25cm} | p{0.25cm} |}
                \hline
                31 & 16 & 45 & 87 & 37 & 99 & 21 & 43 & 10 & 48 \\ \hline
            \end{tabular}
            \begin{tabular}{p{0.25cm} p{0.25cm} p{0.25cm} p{0.25cm} p{0.25cm} p{0.25cm} p{0.25cm} p{0.25cm} p{0.25cm} p{0.25cm}}
                0 & 1 & 2 & 3 & 4 & 5 & 6 & 7 & 8 & 9
            \end{tabular}
            \begin{tabular}{p{0.25cm} p{0.25cm} p{0.25cm} p{0.25cm} p{0.25cm} p{0.25cm} p{0.25cm} p{0.25cm} p{0.25cm} p{0.25cm}}
                & & & & & & & & &
            \end{tabular}
        \end{table}
	\end{center}
    \begin{itemize}[<+->]
        \item Qual o algoritmo para encontrar o número 87?
        \item Dado que o vetor tem 10 posições.
        \item Pode-se ser feito uma busca sequencial.
    \end{itemize}
\end{frame}

\begin{frame}
    \frametitle{Busca Sequencial}
    \begin{center}
        \begin{table}
            \begin{tabular}{| p{0.25cm} | p{0.25cm} | p{0.25cm} | p{0.25cm} | p{0.25cm} | p{0.25cm} | p{0.25cm} | p{0.25cm} | p{0.25cm} | p{0.25cm} |}
                \hline
                31 & 16 & 45 & 87 & 37 & 99 & 21 & 43 & 10 & 48 \\ \hline
            \end{tabular}
            \begin{tabular}{p{0.25cm} p{0.25cm} p{0.25cm} p{0.25cm} p{0.25cm} p{0.25cm} p{0.25cm} p{0.25cm} p{0.25cm} p{0.25cm}}
                0 & 1 & 2 & 3 & 4 & 5 & 6 & 7 & 8 & 9
            \end{tabular}
            \begin{tabular}{p{0.25cm} p{0.25cm} p{0.25cm} p{0.25cm} p{0.25cm} p{0.25cm} p{0.25cm} p{0.25cm} p{0.25cm} p{0.25cm}}
                \color{red}{$\uparrow$} & & & & & & & & &
            \end{tabular}
        \end{table}

        $31 == 87$ ? Não
	\end{center}
\end{frame}

\begin{frame}
    \frametitle{Busca Sequencial}
    \begin{center}
        \begin{table}
            \begin{tabular}{| p{0.25cm} | p{0.25cm} | p{0.25cm} | p{0.25cm} | p{0.25cm} | p{0.25cm} | p{0.25cm} | p{0.25cm} | p{0.25cm} | p{0.25cm} |}
                \hline
                31 & 16 & 45 & 87 & 37 & 99 & 21 & 43 & 10 & 48 \\ \hline
            \end{tabular}
            \begin{tabular}{p{0.25cm} p{0.25cm} p{0.25cm} p{0.25cm} p{0.25cm} p{0.25cm} p{0.25cm} p{0.25cm} p{0.25cm} p{0.25cm}}
                0 & 1 & 2 & 3 & 4 & 5 & 6 & 7 & 8 & 9
            \end{tabular}
            \begin{tabular}{p{0.25cm} p{0.25cm} p{0.25cm} p{0.25cm} p{0.25cm} p{0.25cm} p{0.25cm} p{0.25cm} p{0.25cm} p{0.25cm}}
                 & \color{red}{$\uparrow$} & & & & & & & &
            \end{tabular}
        \end{table}

        $16 == 87$ ? Não
	\end{center}
\end{frame}

\begin{frame}
    \frametitle{Busca Sequencial}
    \begin{center}
        \begin{table}
            \begin{tabular}{| p{0.25cm} | p{0.25cm} | p{0.25cm} | p{0.25cm} | p{0.25cm} | p{0.25cm} | p{0.25cm} | p{0.25cm} | p{0.25cm} | p{0.25cm} |}
                \hline
                31 & 16 & 45 & 87 & 37 & 99 & 21 & 43 & 10 & 48 \\ \hline
            \end{tabular}
            \begin{tabular}{p{0.25cm} p{0.25cm} p{0.25cm} p{0.25cm} p{0.25cm} p{0.25cm} p{0.25cm} p{0.25cm} p{0.25cm} p{0.25cm}}
                0 & 1 & 2 & 3 & 4 & 5 & 6 & 7 & 8 & 9
            \end{tabular}
            \begin{tabular}{p{0.25cm} p{0.25cm} p{0.25cm} p{0.25cm} p{0.25cm} p{0.25cm} p{0.25cm} p{0.25cm} p{0.25cm} p{0.25cm}}
                & & \color{red}{$\uparrow$} & & & & & & &
            \end{tabular}
        \end{table}

        $45 == 87$ ? Não
	\end{center}
\end{frame}

\begin{frame}
    \frametitle{Busca Sequencial}
    \begin{center}
        \begin{table}
            \begin{tabular}{| p{0.25cm} | p{0.25cm} | p{0.25cm} | p{0.25cm} | p{0.25cm} | p{0.25cm} | p{0.25cm} | p{0.25cm} | p{0.25cm} | p{0.25cm} |}
                \hline
                31 & 16 & 45 & 87 & 37 & 99 & 21 & 43 & 10 & 48 \\ \hline
            \end{tabular}
            \begin{tabular}{p{0.25cm} p{0.25cm} p{0.25cm} p{0.25cm} p{0.25cm} p{0.25cm} p{0.25cm} p{0.25cm} p{0.25cm} p{0.25cm}}
                0 & 1 & 2 & 3 & 4 & 5 & 6 & 7 & 8 & 9
            \end{tabular}
            \begin{tabular}{p{0.25cm} p{0.25cm} p{0.25cm} p{0.25cm} p{0.25cm} p{0.25cm} p{0.25cm} p{0.25cm} p{0.25cm} p{0.25cm}}
                & & & \color{red}{$\uparrow$} & & & & & &
            \end{tabular}
        \end{table}

        $87 == 87$ ? Sim

        \Large Encontrado!
	\end{center}
\end{frame}

\begin{frame}
	\frametitle{Busca Sequencial}
    \centering
    \lstinputlisting[language=Java]{src/sequencial.java}
\end{frame}

\begin{frame}
    \begin{center}
        \Large Existe uma maneira mais eficiente de buscar um elemento?
	\end{center}
\end{frame}

\section{Busca Binária}

\begin{frame}
    \frametitle{Busca Binária}
    \begin{itemize}
        \item O vetor precisa estar ordenado, caso contrário o método não funciona.
        \item Esse é o algoritmo de Busca Binária
    \end{itemize}
\end{frame}

\begin{frame}
    \frametitle{Busca Binária}
    \begin{center}
        \begin{table}
            \begin{tabular}{| p{0.25cm} | p{0.25cm} | p{0.25cm} | p{0.25cm} | p{0.25cm} | p{0.25cm} | p{0.25cm} | p{0.25cm} | p{0.25cm} | p{0.25cm} |}
                \hline
                10 & 16 & 21 & 31 & 37 & 43 & 45 & 48 & 87 & 99 \\ \hline
            \end{tabular}
            \begin{tabular}{p{0.25cm} p{0.25cm} p{0.25cm} p{0.25cm} p{0.25cm} p{0.25cm} p{0.25cm} p{0.25cm} p{0.25cm} p{0.25cm}}
                0 & 1 & 2 & 3 & 4 & 5 & 6 & 7 & 8 & 9
            \end{tabular}
            \begin{tabular}{p{0.25cm} p{0.25cm} p{0.25cm} p{0.25cm} p{0.25cm} p{0.25cm} p{0.25cm} p{0.25cm} p{0.25cm} p{0.25cm}}
                & & & & & & & & &
            \end{tabular}
        \end{table}
	\end{center}
    \begin{itemize}[<+->]
        \item Agora com o vetor ordenado.
        \item Qual o algoritmo para encontrar o número 87?
        \item Dado que o vetor tem 10 posições.
    \end{itemize}
\end{frame}

\begin{frame}
    \frametitle{Busca Binária}
    \begin{center}
        \begin{table}
            \begin{tabular}{| p{0.25cm} | p{0.25cm} | p{0.25cm} | p{0.25cm} | p{0.25cm} | p{0.25cm} | p{0.25cm} | p{0.25cm} | p{0.25cm} | p{0.25cm} |}
                \hline
                10 & 16 & 21 & 31 & 37 & 43 & 45 & 48 & 87 & 99 \\ \hline
            \end{tabular}
            \begin{tabular}{p{0.25cm} p{0.25cm} p{0.25cm} p{0.25cm} p{0.25cm} p{0.25cm} p{0.25cm} p{0.25cm} p{0.25cm} p{0.25cm}}
                0 & 1 & 2 & 3 & 4 & 5 & 6 & 7 & 8 & 9
            \end{tabular}
            \begin{tabular}{p{0.25cm} p{0.25cm} p{0.25cm} p{0.25cm} p{0.25cm} p{0.25cm} p{0.25cm} p{0.25cm} p{0.25cm} p{0.25cm}}
                & & & & & & & & &
            \end{tabular}
        \end{table}
	\end{center}
    \begin{itemize}[<+->]
        \item Primeiro armazena o valor da posição inicial (inicio = 0)
        \item Depois, armazena o valor da posição final (fim = 9)
        \item Encontrar o valor de posição central (centro = (inicio+9)/2)
        \item Compare se o valor armazenado com o procurado:
            \begin{itemize}
                \item Caso encontrado, retorne a posicao
                \item Caso valor procurado seja maior inicio = centro + 1
                \item Caso valor procurado seja menor fim = centro - 1
            \end{itemize}
    \end{itemize}
\end{frame}

\begin{frame}
    \frametitle{Busca Binária}
    \begin{center}
        \begin{table}
            \begin{tabular}{| p{0.25cm} | p{0.25cm} | p{0.25cm} | p{0.25cm} | p{0.25cm} | p{0.25cm} | p{0.25cm} | p{0.25cm} | p{0.25cm} | p{0.25cm} |}
                \hline
                10 & 16 & 21 & 31 & 37 & 43 & 45 & 48 & 87 & 99 \\ \hline
            \end{tabular}
            \begin{tabular}{p{0.25cm} p{0.25cm} p{0.25cm} p{0.25cm} p{0.25cm} p{0.25cm} p{0.25cm} p{0.25cm} p{0.25cm} p{0.25cm}}
                0 & 1 & 2 & 3 & 4 & 5 & 6 & 7 & 8 & 9
            \end{tabular}
            \begin{tabular}{p{0.25cm} p{0.25cm} p{0.25cm} p{0.25cm} p{0.25cm} p{0.25cm} p{0.25cm} p{0.25cm} p{0.25cm} p{0.25cm}}
                & & & & \color{red}{$\uparrow$} & & & & &
            \end{tabular}
        \end{table}
	\end{center}
    \begin{itemize}[<+->]
        \item $centro = (0 + 9) / 2 \rightarrow centro = 4.5 \rightarrow centro = 4$
        \item $37 == 87$ ? Não
        \item $37 < 87$ ? Sim $\rightarrow inicio = centro + 1 \rightarrow inicio = 5$
    \end{itemize}
\end{frame}

\begin{frame}
    \frametitle{Busca Binária}
    \begin{center}
        \begin{table}
            \begin{tabular}{| p{0.25cm} | p{0.25cm} | p{0.25cm} | p{0.25cm} | p{0.25cm} | p{0.25cm} | p{0.25cm} | p{0.25cm} | p{0.25cm} | p{0.25cm} |}
                \hline
                10 & 16 & 21 & 31 & 37 & 43 & 45 & 48 & 87 & 99 \\ \hline
            \end{tabular}
            \begin{tabular}{p{0.25cm} p{0.25cm} p{0.25cm} p{0.25cm} p{0.25cm} p{0.25cm} p{0.25cm} p{0.25cm} p{0.25cm} p{0.25cm}}
                0 & 1 & 2 & 3 & 4 & 5 & 6 & 7 & 8 & 9
            \end{tabular}
            \begin{tabular}{p{0.25cm} p{0.25cm} p{0.25cm} p{0.25cm} p{0.25cm} p{0.25cm} p{0.25cm} p{0.25cm} p{0.25cm} p{0.25cm}}
                & & & & \color{blue}{$\uparrow$} & & & \color{red}{$\uparrow$} & &
            \end{tabular}
        \end{table}
	\end{center}
    \begin{itemize}[<+->]
        \item $centro = (5 + 9) / 2 \rightarrow centro = 7$
        \item $48 == 87$ ? Não
        \item $48 < 87$ ? Sim $\rightarrow inicio = centro + 1 \rightarrow inicio = 8$
    \end{itemize}
\end{frame}

\begin{frame}
    \frametitle{Busca Binária}
    \begin{center}
        \begin{table}
            \begin{tabular}{| p{0.25cm} | p{0.25cm} | p{0.25cm} | p{0.25cm} | p{0.25cm} | p{0.25cm} | p{0.25cm} | p{0.25cm} | p{0.25cm} | p{0.25cm} |}
                \hline
                10 & 16 & 21 & 31 & 37 & 43 & 45 & 48 & 87 & 99 \\ \hline
            \end{tabular}
            \begin{tabular}{p{0.25cm} p{0.25cm} p{0.25cm} p{0.25cm} p{0.25cm} p{0.25cm} p{0.25cm} p{0.25cm} p{0.25cm} p{0.25cm}}
                0 & 1 & 2 & 3 & 4 & 5 & 6 & 7 & 8 & 9
            \end{tabular}
            \begin{tabular}{p{0.25cm} p{0.25cm} p{0.25cm} p{0.25cm} p{0.25cm} p{0.25cm} p{0.25cm} p{0.25cm} p{0.25cm} p{0.25cm}}
                & & & & \color{blue}{$\uparrow$} & & & \color{blue}{$\uparrow$} & \color{red}{$\uparrow$} &
            \end{tabular}
        \end{table}
	\end{center}
    \begin{itemize}[<+->]
        \item $centro = (8 + 9) / 2 \rightarrow centro = 8.5 \rightarrow centro = 8$
        \item $87 == 87$ ? Sim
    \end{itemize}
\end{frame}

\begin{frame}
    \frametitle{Busca Binária}
    \begin{center}
        \begin{table}
            \begin{tabular}{| p{0.25cm} | p{0.25cm} | p{0.25cm} | p{0.25cm} | p{0.25cm} | p{0.25cm} | p{0.25cm} | p{0.25cm} | p{0.25cm} | p{0.25cm} |}
                \hline
                10 & 16 & 21 & 31 & 37 & 43 & 45 & 48 & 87 & 99 \\ \hline
            \end{tabular}
            \begin{tabular}{p{0.25cm} p{0.25cm} p{0.25cm} p{0.25cm} p{0.25cm} p{0.25cm} p{0.25cm} p{0.25cm} p{0.25cm} p{0.25cm}}
                0 & 1 & 2 & 3 & 4 & 5 & 6 & 7 & 8 & 9
            \end{tabular}
            \begin{tabular}{p{0.25cm} p{0.25cm} p{0.25cm} p{0.25cm} p{0.25cm} p{0.25cm} p{0.25cm} p{0.25cm} p{0.25cm} p{0.25cm}}
                & & & & \color{blue}{$\uparrow$} & & & \color{blue}{$\uparrow$} & \color{red}{$\uparrow$} &
            \end{tabular}
        \end{table}
	\end{center}
    \begin{itemize}
        \item $centro = (8 + 9) / 2 \rightarrow centro = 8.5 \rightarrow centro = 8$
        \item $87 == 87$ ? Sim
    \end{itemize}
    \begin{center}
        \Large Encontrado
    \end{center}
\end{frame}

\begin{frame}
	\frametitle{Busca Binária}
    \centering
    \lstinputlisting[language=Java]{src/binaria.java}
\end{frame}

\section{Busca Sequencial vs. Busca Binária}

\begin{frame}
	\frametitle{Busca Sequencial vs. Busca Binária}
    \begin{itemize}
        \item Busca Sequencial é $O(n)$
        \item Busca Binária é $O(\log_2 n)$
        \item Busca Binária é mais rápida na Sequencial, porém precisa dos dados ordenados
        \item Para ordernar os dados existem vários métodos
    \end{itemize}
\end{frame}

\section{BubbleSort}

\begin{frame}
	\frametitle{BubbleSort}
    \begin{itemize}[<+->]
        \item Existem diversos tipos de métodos para ordenar os dados
        \item A complexidade deles tem a variação de $O(n\log_2 n)$ até $O(n^2)$
        \item O mais fácil de implementar é o BubbleSort
        \item Porém a complexidade dele é $O(n^2)$
    \end{itemize}
\end{frame}

\begin{frame}
    \frametitle{BubbleSort}
    \begin{center}
        \LARGE{Vamos ordernar este vetor!}
    \end{center}
    \begin{center}
        \begin{table}
            \begin{tabular}{| p{0.25cm} | p{0.25cm} | p{0.25cm} | p{0.25cm} | p{0.25cm} | p{0.25cm} | p{0.25cm} | p{0.25cm} | p{0.25cm} | p{0.25cm} |}
                \hline
                31 & 16 & 45 & 87 & 37 & 99 & 21 & 43 & 10 & 48 \\ \hline
            \end{tabular}
            \begin{tabular}{p{0.25cm} p{0.25cm} p{0.25cm} p{0.25cm} p{0.25cm} p{0.25cm} p{0.25cm} p{0.25cm} p{0.25cm} p{0.25cm}}
                0 & 1 & 2 & 3 & 4 & 5 & 6 & 7 & 8 & 9
            \end{tabular}
            \begin{tabular}{p{0.25cm} p{0.25cm} p{0.25cm} p{0.25cm} p{0.25cm} p{0.25cm} p{0.25cm} p{0.25cm} p{0.25cm} p{0.25cm}}
                & & & & & & & & &
            \end{tabular}
        \end{table}
	\end{center}
\end{frame}

\begin{frame}
    \frametitle{BubbleSort}
    \begin{center}
        \begin{table}
            \begin{tabular}{| p{0.25cm} | p{0.25cm} | p{0.25cm} | p{0.25cm} | p{0.25cm} | p{0.25cm} | p{0.25cm} | p{0.25cm} | p{0.25cm} | p{0.25cm} |}
                \hline
                31 & 16 & 45 & 87 & 37 & 99 & 21 & 43 & 10 & 48 \\ \hline
            \end{tabular}
            \begin{tabular}{p{0.25cm} p{0.25cm} p{0.25cm} p{0.25cm} p{0.25cm} p{0.25cm} p{0.25cm} p{0.25cm} p{0.25cm} p{0.25cm}}
                0 & 1 & 2 & 3 & 4 & 5 & 6 & 7 & 8 & 9
            \end{tabular}
            \begin{tabular}{p{0.25cm} p{0.25cm} p{0.25cm} p{0.25cm} p{0.25cm} p{0.25cm} p{0.25cm} p{0.25cm} p{0.25cm} p{0.25cm}}
                \color{red}{$\uparrow$} & \color{blue}{$\uparrow$} & & & & & & & &
            \end{tabular}
        \end{table}
	\end{center}
    \begin{itemize}[<+->]
        \item $31 > 16$ ? Sim
        \item Então troca
    \end{itemize}
\end{frame}

\begin{frame}
    \frametitle{BubbleSort}
    \begin{center}
        \begin{table}
            \begin{tabular}{| p{0.25cm} | p{0.25cm} | p{0.25cm} | p{0.25cm} | p{0.25cm} | p{0.25cm} | p{0.25cm} | p{0.25cm} | p{0.25cm} | p{0.25cm} |}
                \hline
                16 & 31 & 45 & 87 & 37 & 99 & 21 & 43 & 10 & 48 \\ \hline
            \end{tabular}
            \begin{tabular}{p{0.25cm} p{0.25cm} p{0.25cm} p{0.25cm} p{0.25cm} p{0.25cm} p{0.25cm} p{0.25cm} p{0.25cm} p{0.25cm}}
                0 & 1 & 2 & 3 & 4 & 5 & 6 & 7 & 8 & 9
            \end{tabular}
            \begin{tabular}{p{0.25cm} p{0.25cm} p{0.25cm} p{0.25cm} p{0.25cm} p{0.25cm} p{0.25cm} p{0.25cm} p{0.25cm} p{0.25cm}}
                & \color{blue}{$\uparrow$} & \color{red}{$\uparrow$} & & & & & & &
            \end{tabular}
        \end{table}
	\end{center}
    \begin{itemize}[<+->]
        \item $31 > 45$ ? Não
        \item Então não troca
    \end{itemize}
\end{frame}

\begin{frame}
    \frametitle{BubbleSort}
    \begin{center}
        \begin{table}
            \begin{tabular}{| p{0.25cm} | p{0.25cm} | p{0.25cm} | p{0.25cm} | p{0.25cm} | p{0.25cm} | p{0.25cm} | p{0.25cm} | p{0.25cm} | p{0.25cm} |}
                \hline
                16 & 31 & 45 & 87 & 37 & 99 & 21 & 43 & 10 & 48 \\ \hline
            \end{tabular}
            \begin{tabular}{p{0.25cm} p{0.25cm} p{0.25cm} p{0.25cm} p{0.25cm} p{0.25cm} p{0.25cm} p{0.25cm} p{0.25cm} p{0.25cm}}
                0 & 1 & 2 & 3 & 4 & 5 & 6 & 7 & 8 & 9
            \end{tabular}
            \begin{tabular}{p{0.25cm} p{0.25cm} p{0.25cm} p{0.25cm} p{0.25cm} p{0.25cm} p{0.25cm} p{0.25cm} p{0.25cm} p{0.25cm}}
                & & \color{red}{$\uparrow$} & \color{blue}{$\uparrow$} & & & & & &
            \end{tabular}
        \end{table}
	\end{center}
    \begin{itemize}[<+->]
        \item $45 > 87$ ? Não
        \item Então não troca
    \end{itemize}
\end{frame}

\begin{frame}
    \frametitle{BubbleSort}
    \begin{center}
        \begin{table}
            \begin{tabular}{| p{0.25cm} | p{0.25cm} | p{0.25cm} | p{0.25cm} | p{0.25cm} | p{0.25cm} | p{0.25cm} | p{0.25cm} | p{0.25cm} | p{0.25cm} |}
                \hline
                16 & 31 & 45 & 87 & 37 & 99 & 21 & 43 & 10 & 48 \\ \hline
            \end{tabular}
            \begin{tabular}{p{0.25cm} p{0.25cm} p{0.25cm} p{0.25cm} p{0.25cm} p{0.25cm} p{0.25cm} p{0.25cm} p{0.25cm} p{0.25cm}}
                0 & 1 & 2 & 3 & 4 & 5 & 6 & 7 & 8 & 9
            \end{tabular}
            \begin{tabular}{p{0.25cm} p{0.25cm} p{0.25cm} p{0.25cm} p{0.25cm} p{0.25cm} p{0.25cm} p{0.25cm} p{0.25cm} p{0.25cm}}
                & & & \color{blue}{$\uparrow$} & \color{red}{$\uparrow$} & & & & &
            \end{tabular}
        \end{table}
	\end{center}
    \begin{itemize}[<+->]
        \item $87 > 37$ ? Sim
        \item Então troca
    \end{itemize}
\end{frame}

\begin{frame}
    \frametitle{BubbleSort}
    \begin{center}
        \begin{table}
            \begin{tabular}{| p{0.25cm} | p{0.25cm} | p{0.25cm} | p{0.25cm} | p{0.25cm} | p{0.25cm} | p{0.25cm} | p{0.25cm} | p{0.25cm} | p{0.25cm} |}
                \hline
                16 & 31 & 45 & 37 & 87 & 99 & 21 & 43 & 10 & 48 \\ \hline
            \end{tabular}
            \begin{tabular}{p{0.25cm} p{0.25cm} p{0.25cm} p{0.25cm} p{0.25cm} p{0.25cm} p{0.25cm} p{0.25cm} p{0.25cm} p{0.25cm}}
                0 & 1 & 2 & 3 & 4 & 5 & 6 & 7 & 8 & 9
            \end{tabular}
            \begin{tabular}{p{0.25cm} p{0.25cm} p{0.25cm} p{0.25cm} p{0.25cm} p{0.25cm} p{0.25cm} p{0.25cm} p{0.25cm} p{0.25cm}}
                & & & & \color{red}{$\uparrow$} & \color{blue}{$\uparrow$} & & & &
            \end{tabular}
        \end{table}
	\end{center}
    \begin{itemize}[<+->]
        \item $87 > 99$ ? Não
        \item Então não troca
    \end{itemize}
\end{frame}

\begin{frame}
    \frametitle{BubbleSort}
    \begin{center}
        \begin{table}
            \begin{tabular}{| p{0.25cm} | p{0.25cm} | p{0.25cm} | p{0.25cm} | p{0.25cm} | p{0.25cm} | p{0.25cm} | p{0.25cm} | p{0.25cm} | p{0.25cm} |}
                \hline
                16 & 31 & 45 & 37 & 87 & 99 & 21 & 43 & 10 & 48 \\ \hline
            \end{tabular}
            \begin{tabular}{p{0.25cm} p{0.25cm} p{0.25cm} p{0.25cm} p{0.25cm} p{0.25cm} p{0.25cm} p{0.25cm} p{0.25cm} p{0.25cm}}
                0 & 1 & 2 & 3 & 4 & 5 & 6 & 7 & 8 & 9
            \end{tabular}
            \begin{tabular}{p{0.25cm} p{0.25cm} p{0.25cm} p{0.25cm} p{0.25cm} p{0.25cm} p{0.25cm} p{0.25cm} p{0.25cm} p{0.25cm}}
                & & & & & \color{blue}{$\uparrow$} & \color{red}{$\uparrow$} & & &
            \end{tabular}
        \end{table}
	\end{center}
    \begin{itemize}[<+->]
        \item $99 > 21$ ? Sim
        \item Então troca
    \end{itemize}
\end{frame}

\begin{frame}
    \frametitle{BubbleSort}
    \begin{center}
        \begin{table}
            \begin{tabular}{| p{0.25cm} | p{0.25cm} | p{0.25cm} | p{0.25cm} | p{0.25cm} | p{0.25cm} | p{0.25cm} | p{0.25cm} | p{0.25cm} | p{0.25cm} |}
                \hline
                16 & 31 & 45 & 37 & 87 & 21 & 99 & 43 & 10 & 48 \\ \hline
            \end{tabular}
            \begin{tabular}{p{0.25cm} p{0.25cm} p{0.25cm} p{0.25cm} p{0.25cm} p{0.25cm} p{0.25cm} p{0.25cm} p{0.25cm} p{0.25cm}}
                0 & 1 & 2 & 3 & 4 & 5 & 6 & 7 & 8 & 9
            \end{tabular}
            \begin{tabular}{p{0.25cm} p{0.25cm} p{0.25cm} p{0.25cm} p{0.25cm} p{0.25cm} p{0.25cm} p{0.25cm} p{0.25cm} p{0.25cm}}
                & & & & & & \color{red}{$\uparrow$} & \color{blue}{$\uparrow$} & &
            \end{tabular}
        \end{table}
	\end{center}
    \begin{itemize}[<+->]
        \item $99 > 43$ ? Sim
        \item Então troca
    \end{itemize}
\end{frame}

\begin{frame}
    \frametitle{BubbleSort}
    \begin{center}
        \begin{table}
            \begin{tabular}{| p{0.25cm} | p{0.25cm} | p{0.25cm} | p{0.25cm} | p{0.25cm} | p{0.25cm} | p{0.25cm} | p{0.25cm} | p{0.25cm} | p{0.25cm} |}
                \hline
                16 & 31 & 45 & 37 & 87 & 21 & 43 & 99 & 10 & 48 \\ \hline
            \end{tabular}
            \begin{tabular}{p{0.25cm} p{0.25cm} p{0.25cm} p{0.25cm} p{0.25cm} p{0.25cm} p{0.25cm} p{0.25cm} p{0.25cm} p{0.25cm}}
                0 & 1 & 2 & 3 & 4 & 5 & 6 & 7 & 8 & 9
            \end{tabular}
            \begin{tabular}{p{0.25cm} p{0.25cm} p{0.25cm} p{0.25cm} p{0.25cm} p{0.25cm} p{0.25cm} p{0.25cm} p{0.25cm} p{0.25cm}}
                & & & & & & & \color{blue}{$\uparrow$} & \color{red}{$\uparrow$} &
            \end{tabular}
        \end{table}
	\end{center}
    \begin{itemize}[<+->]
        \item $99 > 10$ ? Sim
        \item Então troca
    \end{itemize}
\end{frame}

\begin{frame}
    \frametitle{BubbleSort}
    \begin{center}
        \begin{table}
            \begin{tabular}{| p{0.25cm} | p{0.25cm} | p{0.25cm} | p{0.25cm} | p{0.25cm} | p{0.25cm} | p{0.25cm} | p{0.25cm} | p{0.25cm} | p{0.25cm} |}
                \hline
                16 & 31 & 45 & 37 & 87 & 21 & 43 & 10 & 99 & 48 \\ \hline
            \end{tabular}
            \begin{tabular}{p{0.25cm} p{0.25cm} p{0.25cm} p{0.25cm} p{0.25cm} p{0.25cm} p{0.25cm} p{0.25cm} p{0.25cm} p{0.25cm}}
                0 & 1 & 2 & 3 & 4 & 5 & 6 & 7 & 8 & 9
            \end{tabular}
            \begin{tabular}{p{0.25cm} p{0.25cm} p{0.25cm} p{0.25cm} p{0.25cm} p{0.25cm} p{0.25cm} p{0.25cm} p{0.25cm} p{0.25cm}}
                & & & & & & & & \color{red}{$\uparrow$} & \color{blue}{$\uparrow$}
            \end{tabular}
        \end{table}
	\end{center}
    \begin{itemize}[<+->]
        \item $99 > 48$ ? Sim
        \item Então troca
    \end{itemize}
\end{frame}

\begin{frame}
    \frametitle{BubbleSort}
    \begin{center}
        \begin{table}
            \begin{tabular}{| p{0.25cm} | p{0.25cm} | p{0.25cm} | p{0.25cm} | p{0.25cm} | p{0.25cm} | p{0.25cm} | p{0.25cm} | p{0.25cm} | p{0.25cm} |}
                \hline
                16 & 31 & 45 & 37 & 87 & 21 & 43 & 10 & 48 & 99 \\ \hline
            \end{tabular}
            \begin{tabular}{p{0.25cm} p{0.25cm} p{0.25cm} p{0.25cm} p{0.25cm} p{0.25cm} p{0.25cm} p{0.25cm} p{0.25cm} p{0.25cm}}
                0 & 1 & 2 & 3 & 4 & 5 & 6 & 7 & 8 & 9
            \end{tabular}
            \begin{tabular}{p{0.25cm} p{0.25cm} p{0.25cm} p{0.25cm} p{0.25cm} p{0.25cm} p{0.25cm} p{0.25cm} p{0.25cm} p{0.25cm}}
                & & & & & & & & & \color{red}{$\uparrow$}
            \end{tabular}
        \end{table}
	\end{center}
    \begin{itemize}
        \item $99$ se mantém na última posição do vetor, pois está ordenado
    \end{itemize}
\end{frame}

\begin{frame}
    \frametitle{BubbleSort}
    \begin{itemize}
        \item O processo deve ser repetido até a ordenação completa do vetor
    \end{itemize}
    \begin{center}
        \begin{table}
            \begin{tabular}{| p{0.25cm} | p{0.25cm} | p{0.25cm} | p{0.25cm} | p{0.25cm} | p{0.25cm} | p{0.25cm} | p{0.25cm} | p{0.25cm} | p{0.25cm} |}
                \hline
                10 & 16 & 21 & 31 & 37 & 43 & 45 & 48 & 87 & 99 \\ \hline
            \end{tabular}
            \begin{tabular}{p{0.25cm} p{0.25cm} p{0.25cm} p{0.25cm} p{0.25cm} p{0.25cm} p{0.25cm} p{0.25cm} p{0.25cm} p{0.25cm}}
                0 & 1 & 2 & 3 & 4 & 5 & 6 & 7 & 8 & 9
            \end{tabular}
            \begin{tabular}{p{0.25cm} p{0.25cm} p{0.25cm} p{0.25cm} p{0.25cm} p{0.25cm} p{0.25cm} p{0.25cm} p{0.25cm} p{0.25cm}}
                & & & & & & & & &
            \end{tabular}
        \end{table}
	\end{center}
\end{frame}

\begin{frame}
	\frametitle{BubbleSort}
    \centering
    \lstinputlisting[language=Java]{src/bubblesort.java}
\end{frame}

\section{Exercícios}

\begin{frame}
    \frametitle{Exercícios}
    \begin{enumerate}
        \item Implemente o método de Busca Sequencial.
        \item Implemente o método de Busca Binária.
        \item Implemente o método de ordenação BubbleSort.
        \item Teste todos os algoritmos em um algoritmo principal
    \end{enumerate}
\end{frame}

\section{Referências}

\begin{frame}
    \frametitle{Referências Bibliográficas}
    \begin{enumerate}
        \item Cormen, Thomas H., Charles E. Leiserson, Ronald L. Rivest, and Clifford Stein. ``Introduction to algorithms second edition.'' (2001).
        \item Tamassia, Roberto, and Michael T. Goodrich. ``Estrutura de Dados e Algoritmos em Java.'' Porto Alegre, Ed. Bookman 4 (2007).
        \item Ascencio, Ana Fernanda Gomes, and Graziela Santos de Araújo. ``Estruturas de Dados: algoritmos, análise da complexidade e implementações em JAVA e C/C++.'' São Paulo: Perarson Prentice Halt 3 (2010).
    \end{enumerate}
\end{frame}

\end{document}
